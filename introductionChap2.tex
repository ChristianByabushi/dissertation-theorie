\documentclass[12pt,a4paper]{report}
\usepackage{graphics} 
\usepackage{graphicx}
\usepackage{amsmath}
\usepackage[english]{babel}
\usepackage{geometry}
\usepackage{hyperref}
\usepackage[utf8]{inputenc} 
\usepackage{listings}  
\usepackage{minted} 
\usepackage{hyperref}  
\usepackage{enumitem} 
\usepackage{csquotes}
\usepackage[labelfont=bf]{caption}
%\renewcommand{\thepart}{\roman{part}}
\usemintedstyle{default}
% Define a style for Python code
\lstset{
	language=Python,
	basicstyle=\ttfamily\small, % Font size and style
	keywordstyle=\color{blue}, % Keywords font color
	stringstyle=\color{orange}, % Strings font color
	commentstyle=\color{gray}, % Comments font color
	showstringspaces=false, % Don't show spaces in strings
	breaklines=true, % Wrap long lines
	numbers=left, % Show line numbers
	numberstyle=\tiny\color{gray}, % Line numbers font style
	frame=single, % Add a frame around the code
} 

\geometry{left=2.5cm, right=2.5cm, top=2.5cm, bottom=2.5cm}
\begin{document} 
	\begin{titlepage}
	\begin{center}
		\LARGE{{\textbf{CATHOLIC UNIVERSITY OF BUKAVU}}}\\
		\begin{center}
			\begin{figure}[h]
				\centering
					\includegraphics[width=4cm, height=4cm]{"../../../../Latex Projects/ucb"}
			\end{figure}
			\large{\underline{B.P.285 BUKAVU}}
				\vspace{0.3cm}
		\end{center}
		\hspace*{0.5cm}
		{\large {\huge {\LARGE 	FACULTY OF SCIENCES  \textsf{}}}}\\ 
			{\Large \hspace*{0.7cm} Department of Computer Science}
			\vspace*{0.1cm}
			\setlength{\fboxsep}{4mm}
			\setlength{\fboxrule}{1mm}
			\vspace{0.5 cm}
			\rule{1\textwidth}{3pt}\\
			\vspace{0.18 cm}
			\begin{minipage}[c]{15cm}
				\begin{center}
					\LARGE{\textbf{\textcolor{black}{Development of a messaging application for communication and detection of spam on a mobile operator, case study of Airtel, Vodacom and Orange.}}}
				\end{center}
		\end{minipage}
	\end{center}
	\hspace{3pt}\rule{1\textwidth}{3pt}
	\vspace{0.1cm}
	\begin {minipage}{0.5 \textwidth }	
	\begin{flushright}
		{\large 
			\vspace {0.1cm} 
			\begin{tabbing}					
				\hspace*{1cm} \\
				\\
				\hspace*{2cm} Presented by : \textbf{MURHULA BYABUSHI Christian}  \\
				\hspace{2cm} \textit{Dissertation presented and defended in order to obtain the} \\
				\hspace{2cm} \textit{degree of Bachelor in Computer Science.}\\
				\vspace*{0.3cm}\\
				\hspace*{2cm} Option: Network and Telecommunications\\					
				\hspace*{2cm} Degree: Final year of Bachelor\\ 
				\vspace*{2cm}\\
				\\
				\hspace*{2cm}Supervised by: \textbf{\textit{Hw}. MUGISHO MUSHEGERHA Youen }\\
				\hspace*{2cm}Directed by : \textbf{\textit{PhD}. Elie ZIHINDULA}			
			\end{tabbing}
		}					

	\end{flushright}
\end{minipage}
\begin{center}
\end{center}

\begin{center} 
	\huge{\textbf{Academic year: 2022-2023}}
\end{center}
\end{titlepage} 
	\tableofcontents 
	\listoffigures
	\listoftables
	
	\newpage
	

	\addcontentsline{toc}{chapter}{Introduction} 
	\chapter*{Introduction}
	\section{Context and generalities} 
	With the increasing of use of mobile devices in mobile telecommunication, the number of text messages sent every day has grown exponentially. According to \textit{Statista}, a company that provides market and consumer data on a wide range of topics, including digital media and technology; the number of mobile messages sent worldwide in 2020 reached 3.5 trillion \cite{Statista2020}. 
	In the same case, with the raise of web pages and social media messaging applications like Whatshap, Teelgra, Snapchat Facebook, Instagram and many others, phone users can now send messages  that are only based on text as in former time but also on video, audios which are more chestful comparatively \cite{faklaris2016oh}. \\
	
	For sure, for interacting with his partner more professionally, email message is the most mean used, but not in all cases since in some countries a given SIM Card of a telecommunication provider is used as a bank more than being an communication mean, so the importance to secure the communication lines of a mobile users in these countries.
	Along with the functions and interests that the mobiles messages encompasses in terms of conversing, money sending and receiving, there has been an increase in the number of spam messages that aim to deceive people into providing personal information, sending unwillingly money, menacing to death or taking other actions that benefit the scammer.\\
		
	To address this problem, the development of a messaging application with advanced spam detection capabilities is crucial to set, filter messages and prevent the users. This dissertation focuses specifically on the development of such application for phone users and in general for mobile networks telecommunications technologies.  
	
	%such as : Airtel, Vodacom and Orange. \\
	% Indeed, Airtel, Vodacom, and Orange are major telecommunications operators which mostly function in Democratic Republic of Congo, with a significant market share and importance in online services, including messaging. 
	 
	\section{Problematic} 
	In telecommunication domain, we use mobile devices or phones for sharing \textit{SMS}, Email, chats by using some specific apps. Among all we use specifically \textit{SMS} for personal and professional information sharing \cite{lavanya2582sms}. The SMS stands for Short Message Service, which is a text messaging service for mobile phones and other mobile devices. It allows users to send and receive short messages of up to 160 characters \cite{le2005mobile}. It is also possible to send or receive automatic SMS which are not sent by human, whereas by using web interface or API \cite{jangir2016design}.
	
	Time to time, more persons are receiving messages such as : "You won x amount of money send another amount to withdraw it", "Join me at x area to take your money but pay me the transport ...", "I'm \textit{Sirene} Madam I have money for you", "I have a job for you" and many others fake messages. More of them are reported in this  \href{ https://blog.textedly.com/spam-text-message-examples}{spams examples link}.  
	
	Furthermore, scammers go the point where they can introduce vulnerabilities in messages which exploit a weakness in the SMS messaging system to remotely install spyware on mobile devices commonly called \textit{Simjacker}  \cite{cimpanu2019simjacker}. However, in marketing  almost similar messages are used to sensitize people to by products and services which confuse users wether it is or not a spam message by leading users to ignore important messages or being more hesitant to engage with mobile marketing campaigns \cite{chen2017survey},\cite{leppaniemi2008mobile}.
	
	Considering all above issues caused by spam, what are key technical challenges that could be addressed in messaging system which can effectively facilitate the communication as well as detecting and filtering spam messages ? 
	
	\section{Hypotheses}  
	 According to the Oxford dictionary, hypotheses stands for a statement of the expected relationship between things being studied, which is intended to explain certain facts or observations \cite{park2012dictionary}. An idea to be tested.
	 Hence, using the content-based filtering techniques, which involves analyzing the content of messages and determining wether it is a spam or not would be considered as solution. 
	
	 Firstly this would be done by utilizing the Machine learning algorithms which are : \textbf{Naive Bayes, Logistic Regression, and Supper vector Machines}. All these would be combined by the ensemble methods for making a more predictive model \cite{raschka2017python} including the preparation and classification techniques which avoid biased model \cite{karl1986model}. 
	 
	 Secondly, during the production steps, we should integrate the model inside of the system able to add technically in a blacklist or whitelist suspect users based on the specific probability of being a potential attack.
	 
	 Overall, we will jump from Machine Learning as model (\textit{MLaaM}) which is the output of writing ML algorithms run on data and represent what was learned by the algorithm on training data; to ML Model Software Deployment which encompasses all the activities that make a software system worthy to be used \cite{hadullo2021machine}.
	 
	\section{Delimitation and objectives}  
	\subsection{Delimitation}
	The present work aims to develop a messaging application for communication and detection of spam on a mobile network.
	
	Geographically it focuses on all provinces of Democratic Republic Of Congo(DRC) where mobile phones are used and require techniques for implementation. 
	
	Besides, it does not function effectively across all languages unless the solution model has been specifically trained on those languages. As a result, it is challenging to claim its effectiveness in languages such as Swahili, Lingala, French, or even English. Moreover, achieving optimal performance often necessitates the involvement of a large population.
	
	Indeed, the current work in terms of planning and execution has spanned a duration of nine months : From January to November 2023. 
	\subsection{Objectives}
	This system present 2 types of objectives such as: functional and non functional.
	
	As functional objectives, this system consist of developing a messaging application that can facilitate efficient communication between users; implementing the machine learning models which has the capacity of classifying the messages and preventing spam messages under a certain probability; designing and integrating a user-friendly interface for messaging application;  
		
	For non-functional objectives, it allows the user whose messaging application complies with relevant privacy laws and regulation to protect data information, reducing the attacks and frauds; Increasing the trustfulness of users and mobile services compagnie provider, optimizing the power resources of the user against threats posed by scammers. 
	
	\section{Interest} 
	Personally, this paper has allowed the author to gain knowledge and more experience in the field of mobile networks and messaging applications. 

	Socially, the developed system contributes to facilitating communication and reducing the impact  of spam messages, which can be annoying and stressful for citizens.
	
	Economically, this system of detection helps to save business server time and resources by filtering out spam messages allowing for more targeted marketing efforts. 
	
	Scientifically, the research achieved contributes to the advancement of the science in the domain of Mobile Networks, SMS messaging, and Machine Learning data processing and classification.
	
	\section{Research Methodology}
	Throughout this paper, the research methodology will be used to guide the study towards achieving its objectives. The research will adopt a descriptive research design to describe the development of a messaging application for communication and detection of spam on a mobile network. The study will focus on both qualitative and quantitative research methods \cite{creswell2014research}. The qualitative method will involve a literature review, interviews and analysis of collected messages. while quantitative method will focus on the development and testing of the messaging application.
	
	The research will be conducted in two phases. The first phase will involve data collection through a survey questionnaire that can be completed on website, or can be directly provided to the web interface (API) by the mobile phone users for collecting their experience with messages and especially spams. Thus, the data collected will be analyzed using descriptive statistics \cite{bluman2017elementary} to identify the common types of spam messages and the frequency of occurrence, languages inside, and other attributes.  
	
	The second phase will involve the development of the messaging application using the data collected from the survey and the analysis of existing messaging applications. The development of the application will be guided by the principles of agile software development by using Python (Django framework) for \textit{Back-end} and HTML,CSS and JavaScript for \textit{front-end}. Then,the application will  integrate the use of machine learning models, including Naive Bayes, Logistic regression, and Support Vector Machine.
	
	The evaluation of the messaging application will be conducted using both quantitative and qualitative methods. The qualitative evaluation will involve the measurement of the application's accuracy and efficiency in detecting and filtering spam messages, on the other hand the qualitative evaluation will involve a user study to determine the usability and user experience of the application.
	
	\section{Work Plan (or Work Subdivision)}
	The work plan of this dissertation is divided into four parts. The first is the introduction, which provides a background information on the research problem.
	The second part consist of a situation analysis and assessment while the third part focuses on literature review and explanations on the methodology. Then the fourth part presents the practical result of this work. Finally the conclusion part summarizes the key findings and contributions of the study and presents limitations and provide recommendations for future research. 
		
	\chapter{Situation analysis and assessment on mobile phones}
	\section{Introduction}
	In this chapter, we will focus on various aspects that enhance the comprehensiveness and practicality of this dissertation. It includes explanations of mobile messaging architecture, machine learning models, and spam messages in mobile world. Additionally, it provides an analysis of the architecture used by network operators, highlighting both positive and negative aspects of their approach to message handling.
	\section{Presentation of the working framework and definition of key concepts}
	\subsection{Definition of key concepts}
	\begin{enumerate}[label=\alph*)]
		\item SMS(Short Message Service) :\\
		The Short Message Service is a basic service allowing the exchange of short text messages between subscribers \cite{mobilemessaging}. For supporting virtually all mobile devices, SMS is considered as a universal means of communication that enables users to communicate and function even though all users are not active simultaneously (asynchronous communication).
		\item Enhanced Messaging Service (EMS) : \\
		EMS has been created to allow the transmission of richer and more advanced messages. Unlike traditional SMS, EMS accepts not only text messages but also audios, melodies, and animations \cite{le2005mobile}.
	    
	    \item MMS (Multimedia Messaging Service):\\
	    MSS has been developed to facilitate the transmission of rich multimedia content in mobile messaging. Unlike SMS and EMS, MMS enables users to send not only text messages but also various types of multimedia files such as images, videos, audio recordings and even slideshows \cite{le2005mobile}.
	    \newpage
		\item Spam message: \\
		A spam message is understood as an unsolicited or undesired messages received on mobile phones which constitutes veritable nuisance to the mobile subscribers \cite{shafi2017review}. Clearly, this message can be sent with the intention of gaining financial benefits, collecting personal or organizational information such as security numbers, credit card details, or login credentials, and soliciting money by making false promises of future benefits or rewards that do not materialize. 
		
		\item Networks operators:
		The  networks operators refers to companies or organizations that provide and manage telecommunication networks. These operators own and operate the infrastructure, such as mobile networks, fixed-line networks, or internet service provider (ISP) networks, that enable the transmission of user's information to another user of the network \cite{ghezzi2015strategy} 
		
		\item Artificial Intelligence (AI) :
		 AI refers to the field of computer science that focuses on creating intelligent machines or systems that can perform tasks that would typically require human intelligence. For being practical, it encompasses algorithms, models and technologies that enable computers and machines to simulate human like cognitive processes such as learning, reasoning, problem-solving, perception and language understanding.  
		 
		 \item ML (Machine Learning)  : 
		 Machine learning is a subfield of Artificial Intelligence that focus on the development of algorithms and models that enable computers to learn from data and make decisions or predictions without being explicitly programmed\cite{smola2010introduction}. Clearly, when data is labeled during the training, we refer to it as supervised model. If contrast, we the data is unlabeled and the model must discover patterns and relationships itself, it is an unsupervised model.   
		 
		 \item NLP (Natural Language Process):
		 NLP is a subfield of Machine Learning that studies the human language and combing techniques from statistics, linguistics, life-hoods for making sentiment analysis, text classification, machine translation, question answering and text generation in a way that it can be understood computationally \cite{cambria2014jumping}. 
		  \end{enumerate}
		\subsection{Presentation of the working framework}  
		In the eastern party of DRC (South Kivu- and North Kivu) the usage of mobile phones has become more common, transforming communication and connectivity in the region. 
	The DRC itself is large country, covering over 2,345,000 square kilometers with the eastern provinces of North and South Kivu spanning approximately 59,483 and 65,070 square kilometers respectively \cite{giswatch2018}. According to recent statistics from \textit{GlobalEdge} \footnote{ \href{https://globaledge.msu.edu/}{GlobalEdge}  : Created in 1994 by the International Business Center and the Eli Broad College of Business at Michigan State University (IBC), globalEDGE™ is a knowledge web-portal that connects international business professionals worldwide to a wealth of information, insights, and learning resources on global business activities}, an American company, around 95 million people were living in the DRC in 2022 \cite{monusco2015}, of which approximately 46.9\% had active mobile phones based on \textit{GSM} \footnote{\href{https://www.gsma.com/aboutus/}{GSMA}  (Global System Communications Association) : An industry Organization which represents the interests of mobile network operators worldwide created in 1982 to ease cooperation between countries deploying \textit{GSM} (Global System fo Mobile) technology.} research. \\ 
		
		
		In this context, it is observed that more people in cities use mobile phones compared to those in villages, primarily due to limited accessibility. A research study conducted by \textit{Target Canibet} \footnote{\href{https://www.target-sarl.cd/fr/content/etude-sur-la-telephonie-mobile-en-rdc}{Target Canibet: Reseach \& Consulting Group working in DRC. \url{https://www.target-sarl.cd/fr/content/etude-sur-la-telephonie-mobile-en-rdc} }}in 2015 focused on mobile connections in DRC cities including Bukavu, Goma, Kinshasa, Lubumbashi, and Matadi, found that among 1,000 people surveyed in each city, 9 out of 10 individuals were subscribed to a network operator. However, it was noted that approximately half of them subscribed to two operators, while a quarter subscribed to four operators, and 18\% used the services of a single operator.  
		
		Furthermore, the recent statistics made by \textit{DataReportal} \footnote{DataReportal: A online Company designed to help people and organizations all over the world to find the data, insights, and trends they need to make better informed decisions produced by Simon Kemp, \url{https://datareportal.com/reports/digital-2023-global-overview-report}}  in DRC shows that the mobiles users continues to increase exponentially merely because of new services provided by internet and Telecoms Operators, at the point that since 2021 to 2022, it is has been reported 3.6 million of new users between 2021 to 2022, a report that proves how much mobile phones is inevitable in this last decades.
			
	   \subsection{Network coverage and infrastructure}   
	    In fact, two telecoms services exist in DRC such as : Fixed services (26\%) and Mobile services (74\%). The first one known as landline or wired services, involve the use of physical infrastructure; the second one which is popular is the mobiles services refer to telecommunications services provided through mobile networks.
	    According to the Congolese Regulatory Agency (ARPTC), the DRC has four mobile operators - Vodacom RDC,
	    Airtel Congo, Orange DRC and Africell DRC. Vodacom is the leader in the voice segment, with 35.2\% of the market,
	    followed by Orange (30\%), Airtel (23.9\%) and Africell (10.9\%). In the mobile internet market, Vodacom has 37.44\%, Airtel
	    31.25\%, Orange 28.14\% and Africell 3.17\% \cite{stateInternet2019}. 
	     	        	     	        
	   Additionally, since the 190s, when the DRC witnessed the first installation of operator systems such as Celtel(now Airtel) and Vodacom, followed by Orange and Africell, the telecomunications sector has shown significant market growth, reaching 1 Billion in 2022\$ and expanding at a rate of 21\% per year according to  \textit{GlobalData} \footnote{\href{https://www.globaldata.com/store/report/drc-telecom-operators-market-analysis/}{GlobalData: Expert Company of Analysis, innovatove Solutions}}. 
	   However, this growth necessitates the updating of the infrastructure, which includes various generations of technologies, namely the second generation, third, fourth, and fifth(under development).\\
	   	   
	   In fact, the second generation have been deployed in various territories to enable more efficient \textbf{voice calls, data networks services, and introduce SMS for text messaging}. The infrastructure required for 2G networks includes the following equipments: 1) BTS(Base Transceiver Station) : Transmit and receives signals between mobile devices. 2) MSC(Mobile Switch Controller) : serves as the switching entity that connects calls between mobile devices. 3) BSC (Base Station Controller) :  manages multiples BTSs and controlling radio resources, managing handovers between cells and optimizing network performance, 4) AuC (Authentication Center) responsible for managing subscriber authentication and encryption keys to ensure communication between mobile devices and the network, 5) Home Location Register (HLR) the database that stores subscriber information such as phone numbers, authentication details, and service profiles, 6) Visitor Location Register (VLR) : The VlR is a temporary database that stores information about roaming subscribers within a specific area 7) MS (Mobile Station), including all the technologies used by the  users's handset and has two parts :\textbf{ Firstly, the mobile equipment which contains the radio equipment, the user interface, the processing capability and memory requirements for call signaling, encryption, SMS and the id of the mobile phone(equipment IMEI number). Secondly the Subscriber Identity module (SIM Card),} used in encryption of codes needed to identify the subscriber, storing subscriber's information, locate the user \cite{realWorldNetworks} as (+243 for each congolese number). \\
	   Indeed, all the 2G technologies covers a large distance varying between 1880MHz - 2700 MHz.\\
	      
	   Besides, the third generation appears as revolution, \textbf{allowing multimedia messages, voice calls data, faster data speed}; however it requires a significant upgrade from the previous generation. Thus, the equipment involved in 3G technology includes : 1) BTS (Base station Transceiver) : Which plays the same role as for 2G; 2) Node B: Responsible for handling the radio interface and connecting mobile devices to the core network; 3) Radio Network Controller (RNC) : Controlling the Node B and managing the radio resources 4) Mobile Switching Center (MSC): The MSC is the central switching entity in the network that \textbf{connects calls between mobile devices}; 5) Serving GPRS Support Node (SGSN) : Responsible for managing packet-switched data services services and handling mobility for mobile internet access; 6) Gateway GPRS Support Node (GGSN): It serves as interface between the mobile network and external networks like internet; 7) The Home Location Register (HLR) and Authentication Center(AuC): plays the same role as in 2G; 8) Operations Support System (OSS) : It provides and functionalities for monitoring and managing the 3G network. 
	   Indeed, the 3G is appreciated for enabling higher- speed services and covers different frequency bands depending on countries, ranging between 850Mhz - 1700 Mhz \cite{mishra2007advanced}.\\
	   
	   Additionally, the fourth generation, commonly referred to as LTE(Long-Term Evolution) represents a significant advancement over previous generations in terms of infrastructure and services. This generation introduces higher data speeds, improved capacity, and better perfomance for mobile communication and data services. The upgrades in infrastructure include: 1) BTS and MSCs : These components remain unchanged from the previous generation 2) Evolved Packet Core (EPC)  The EPC is a critical component of the 4G core network architecture which provides the packet-switched backbone that handles data traffic and ensures efficient data delivery between mobile devices and the internet or other networks; 3)  Radio Access Network (RAN): The Ran is responsible for the radio interface between mobile devices and base stations; 4) LTE (Long-Term Evolution) :  is the primary air interface enabling the high data speeds, low latency; 5) Back-haul Network : It connects base stations to the core network and internet infrastructure; 6) Spectrum Allocation: Hands over the mobile operator access to specific radio frequency bands; 7) Network Management System: These systems monitor and manage the 4G network, ensuring its smooth operation, performance optimization, and troubleshooting.
	   However, it's spanning or coverage of 4G networks on frequency bands allowed in each country based on their preferences, ranging from 700MHz to 2600 Mhz. The higher frequency bands generally offer faster data speeds but may have a shorter range, while the lower frequency bands can provide broader coverage but with slightly lower data speeds \cite{mishra2007advanced}.
	\subsection{Mobile Phone Models}  
	Since the mobiles phones are essential tools for communication, there is a wide range of popular mobile widely used by citizens of the DRC. The popular models come from various brands and offer a range of features to cater to different user preferences and needs. Some of the popular mobile phone models in DRC include \textit{ Techno, Itel, Infinix, Samsung, Apple, Huwaei, Itel, HTC, Motorola}. As it can be seen on the figure \ref{fig:mobilevendor}, according  to the  recent statistics conducted by \textit{Statistica},
	Samsung was the market leader in terms of share from January 2018 to November 2020, but in 2022, Tecno has emerged as the market leader.
	\begin{figure}
		\centering 
		\includegraphics[width=18cm, height=11cm]{Images/mobileMarket.png}
		\caption{Market share of mobile device vendors in the Democratic Republic of the Congo from January 2018 to March 2022}
	    \captionsetup{position=top}
		\label{fig:mobilevendor} 
	\end{figure}

Furthermore, all these models provided above sell the telephone following different types which include mobile phones, offering the features such as touchscreen displays, cameras, internet, connectivity, and access to mobile apps; features phones, which are basic mobile phones used for calling and texting; smartphones used by the majority (around 35\% in DRC ), providing access to mobile internet, mobile apps, multimedia messaging, and various productivity tools; Tablets, used for reading and for the same functionalities as smartphones.
\subsection{Mobile usage and prevalence} 
In fact, each phone has its own unique set of characteristics that define its capacity and performance compared to others. Some phones come with specific applications that can be used independently, even without being connected to an operator, such as a camera, calculator, games app, and many others. However, other phones may not have such features.

To access the services provided by the operator, the phone's sim-card must be functioning, recognized by the operator, and capable of sending and receiving communication signals. \textbf{Of course, all these services work only if the phone's battery has sufficient power}.

Moreover, the services that citizen's subscribers benefit from are as follows : 

Firstly, the Internet Access: The Internet services are used to connect people from different nodes. In fact, In accordance with the \textit{WorldBank} \footnote{WorldBank : International Telecommunication Union ( ITU ) World Telecommunication/ICT Indicators Database \url{https://data.worldbank.org/indicator}}  been used by 23\% of the DRC's population in 2020. Nonetheless, it requires payment which proportionally gives mobile data usually expressed in Megabytes. 

Secondly, the Text Messaging (SMS): Even though the internet is the most used for texting, the SMS remains a widely used form of communication, especially in regions with \textbf{limited internet connectivity or among users who prefer simple text messaging or do not have the internet connection}.
Furthermore, with the architecture of GSM(Global System for Mobile Communications) invented in the second generation, sending messages became possible. Nowadays, the web environment has developed application interfaces (API) that connect external systems to operators for sending messages \cite{hassinen2003secure}. One of the platforms that offer these services connects its SMS gateway to the GSM operator, as seen in the case of \textit{Octopush} \footnote{Octopush : SMS platform for businesses connected with their audience} architecture shown in Figure \ref{fig:smsgateway}  

\begin{figure}
	\centering
	\includegraphics[width=0.7\linewidth]{Images/SMSgateway}
	\caption{SMS Gateway Provider Architecture}
	\label{fig:smsgateway} 
\end{figure}
    
Thirdly, the Mobile Banking and Payments : With this services subscriber can make \textbf{financial transactions}; paying bills, transferring money conveniently. 

Fourthly, the Mobile Entertainment: Mobile phones offer a range of entertainment options, from streaming videos and music to mobile gaming. 

And finally, the others Mobiles apps: This party includes the health services, education, social medias applications.
\section{Purposes of spammers in mobile messages}
Most of the time, spammers prefer to promise recipients prizes and then ask for money to claim the offer. They also attack SMS gateways with DoS (Denial of Service) messages \cite{androulidakis2013fimess} whose goal is to overwhelm the system with unnecessary messages. Spammers send advertising and promotional messages based on company objectives, as well as SMS containing fake links or impersonating organizations to deceive recipients into taking certain actions or providing sensitive information \cite{tang2022clues}. Additionally, they may send SMS disguised as surveys to gather personal information for various fraudulent purposes.
\section{Solutions}
To address theses problems, it is necessary to involve various stakeholders, including network operators, app developers, regulatory bodies, and users. Firstly, it is recommended to implement mechanisms at the network level \cite{hao2009detecting} to filter messages and block users involved in spamming. Secondly, users (subscribers) should be educated on how to analyze messages and report any one that is causing disturbances. Thirdly, regulatory measures should be enforced to establish stringent regulations and penalties for spammers and those engaged in fraudulent mobile activities. Fourthly, the development of apps that enable filtering, classification, and reporting in the subscriber side would be beneficial. Fifthly, a website can be set to collect messages whether spam or ham reported by users who have doubts about their legitimacy, and then Machine Learning models can be used for detection purposes. For this, supervised or unsupervised methods can be employed to classify and predicting wether a message is spam or  ham. 
\section{Conclusion}
Overall, with the growth of mobile technologies, subscribers benefit from diversified services including : text messaging, voice calls, mobile baking, entertainment apps, and many others. However, These advancements also bring new challenges, such as the development of spam messages that aim to disturb network subscribers with unwanted or threatening messages. 

In DRC, especially in the eastern party, users face similar issues. This chapter emphasizes methods or techniques that can be used to address this problem and relatively reduce spam. One of the prominent techniques suggested is based on Artificial Intelligence, particularly Machine Learning algorithms.
    
    \chapter{Review of the Literature and description of the approach}
    \section{Introduction}
    This chapter delves into theory, methodologies, and machine learning techniques, including relevant algorithms and their deployment in the suggested solution. It also highlights the the contributions of previous researchers in the field.  
    \section{Revue of the Literature}  
    Numerous researchers have extensively explored the subject of spam detection. Within this domain, some have directed their investigations towards the web environment, while others have delved into realm of mobile	technologies.
    
    Furthermore, these researchers have chosen to investigate the detection of spam across various communication channels, including emails and SMS, encompassing Multimedia SMS (MSS) as well. In the following sections, we comprehensively review the body of work that has been accomplished within this context as follows:
     
	\cite{katankar2010short}. \textbf{Dr.V.M Veena K.Katankar}
	proposed a system that comprises an SMS gateway for transferring SMS messages after they have been stored and encrypted by the web server. This software operates through a web interface. Whenever a client sends a \textit{POST} request, it is received by the web server, which is responsible for encryption or decryption if necessary. Subsequently, the gateway transfers the message as per its designated route. This solution proves to be particularly valuable in mobile banking and organizational marketing systems. Nevertheless, the author encourages other researchers to delve into channel services in communication and advanced encryption techniques.\\
		
	\cite{brown2007sms} In their publication titled \textbf{\textit{Short Message Service}}, \textbf{Brown, Jeff and Shipman} members of IEEE, delve into several significant aspects. They start by exploring th growth of mobile phones and SMS services. They also examine the system architecture of SMS Centers and technologies used for message communication
	
	Furthermore, they shine the spotlight on aggregators and services providers. These are the entities that enable users to send bulk messages, essentially sending messages with a large amount of text to a group of recipients. This includes the interesting capability of converting email messages into SMS.
	
	Moreover, the article highlights that some of these aggregators may choose to collaborate with cellular networks. In this collaborative role, they act as intermediaries, connecting third-party entities that don't have direct relationships with cellular service providers. To achieve this, they employ a \textit{SMPP (Simple Messaging Peer to Peer)} protocols.\\
	
	\cite{medani2011review} \textbf{A. Medani and A. Gani} at the University of Malaysia published a review focused on mobile short message service security issues and techniques towards the solution. This paper shows clearly how a subscriber sends a message to another by respecting certain principles of the \textit{OTA}(\textit{Over-The Air}) structure before being sent to the base station and transmitted to the destination by the \textit{SMSC (SMS Center)}. Indeed, according to the \textit{OTA} every SMS should be secured by the \textit{PKI (Public Key Infrastructure)} which provides an end-to-end transmission security so the message can not be modified by anyone once again. However the PKI decreases the mobile performance as it requires high mobile power capability to apply the process and doesn't assure integrity between all standars. Therefore, to address this issues observed on the GSM systems, this work suggests to deploy for the behalf of clients a \textit{XML Key Management Specification} as a \textit{middleware} (A intermediary system that should work as facilitator between ).   
	\\
	However by this article's exploration 

	
	
    
    
    
	%\bibliographystyle{apacite} % or whatever citation style you choose
	\bibliographystyle{plain}
	\bibliography{bibfile.bib}
	
	\newpage	
	
	
	
	
	
	
	
\end{document}